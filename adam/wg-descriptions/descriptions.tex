\begin{center}
{\Large IETF Working Groups}
\end{center}

\section{ Applications Area}

\subsection{Domain Name System}

\subsection{ 
telnet
}
The TELNET working group is to look at RFC 854, ``Telnet Protocol
Specification'', in light of the last 6 years of technical
advancements, and determine if it is still accurate with how the
TELNET protocol is being used today.  This group will also look at all
the numerous TELNET options, and decide which of them are still
germane to current day implementations of the TELNET protocol.

\begin{itemize}
\item
   Reissue RFC 854 to reflect current knowledge and usage of the
     TELNET protocol.
\item
   Create RFCs for new TELNET options to clarify or fill in any
     missing voids in the current option set.  Specifically:

\begin{itemize}
\item  Environment variable passing
\item  Authentication
\item  Encryption
\item  Compression
\end{itemize}

\item
   o Act as a clearing house for all proposed RFCs that deal with the
     TELNET protocol.

\end{itemize}

\subsection{Network Printing Protocol
}
The Network Printing Working group has the goal of pursuing those
issues which will facilitate the use of printers in an internetworking
environment.  In pursuit of this goal it is expected that we will
present one or more printing protocols to be considered as standards
in the Internet community.

This working group has a number of specific objectives.  To provide a
draft RFC which will describe the LPR protocol.  To describe printing
specific issues on topics currently under discussion within other
working groups (e.g. security and dynamic host configuration) and
present our concerns to those working groups.  Examine printing
protocols which exist or are currently under development and assess
their applicability to Internet wide use, suggesting changes if
necessary.

\newpage

\section{Host and User Services}

\subsection{Distributed File Systems}

Trans- and inter-continental distributed file systems are upon us.
The consequences to the Internet of distributed file system protocol
design and implementation decisions are sufficiently dire that we need
to investigate whether the protocols being deployed are really
suitable for use on the Internet.  There's some evidence that the
opposite is true, e.g., some DFS protocols don't checksum their data,
don't use reasonable MTUs, don't offer credible authentication or
authorization services, don't attempt to avoid congestion, etc.
Accordingly, a working group on DFS has been formed by the IETF. The
WG will attempt to define guidelines for ways that distributed file
systems should make use of the network, and to consider whether any
existing distributed file systems are appropriate candidates for
Internet standardization.  The WG will also take a look at the various
file system protocols to see whether they make data more vulnerable.
This is a problem that is especially severe for Internet users, and a
place where the IETF may wish to exert some influence, both on vendor
offerings and user expectations.


\subsection{Dynamic Host Configuration}

The purpose of this working group is the investigation of network
configuration and reconfiguration management.  We will determine those
configuration functions that can be automated, such as Internet
address assignment, gateway discovery and resource location, and that
which cannot (i.e., those that must be managed by network
administrators).


\subsection{Internet User Population}

To devise and carry out an experiment to estimate the size of the
Internet user population.


\subsection{Network Information Systems Infrastructure}

The NISI WG will explore the requirements for common, shared
Internet-wide network information services. The goal is to develop an
understanding for what is required to implement an information
services ``infrastructure'' for the Internet. This effort will be a
sub- group of the User Services WG and will coordinate closely with
other efforts in the networking community.

\subsection{TCP Large Window Options}

This is a short term, ad hoc, single question working group chartered
to make some progress on the various proposals for TCP in long fat
pipes.

\subsection{User Connectivity}

The User Connectivity working group will study the problem of how to
solve network users' end-to-end connectivity problems.

\subsection{User Documents}

The USER-DOC Working Group will prepare a bibliography of on-line and
hard copy documents/reference materials/training tools addressing
general networking information and ``how to use the Internet''.
(Target audience: those individuals who provide services to end users
and end users themselves.)

\begin{itemize}
\item
   Identify and categorize useful documents/reference
     materials/training tools.
\item
   Publish both an on-line and hard copy of this bibliography.
\item
   Develop and implement procedures to maintain and update the
     bibliography.  Identify an organization or individuals to accept
     responsibility for this effort.
\item
   As a part of the update process, identify new materials for
     inclusion into the active bibliography.
\item
   Set up procedures for periodic review of the biblio by USWG.

\end{itemize}

\subsection{User Services}
}
The User Services Working Group provides a regular forum for people
interested in user services to identify and initiate projects designed
to improve the quality of information available to end-users of the
Internet.  (Note that the actual projects themselves will be handled
by separate groups, such as IETF WGs created to perform certain
projects, or outside organizations such as SIGUCCS.

\begin{itemize}
\item
   Meet on a regular basis to consider projects designed to improve
     services to end-users.  In general, projects should
\begin{itemize}
\item
       clearly address user assistance needs;
\item
       produce an end-result (e.g.  a document, a program plan, etc);
\item
       have a reasonably clear approach to achieving the end-result
         (with an estimated time for completion);
\item
       and not duplicate existing or previous efforts.
\end{itemize}
\item
   Create WGs or other focus groups to carry out projects deemed
     worthy of pursuing.
\item
   Provide a forum in which user services providers can discuss and
     identify common concerns.
\end{itemize}

\newpage




